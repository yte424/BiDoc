\documentclass[11pt,a4paper]{scrartcl} % Basisdokumentenklasse
\usepackage[ngerman]{babel}        % Deutsche Standardbezeichner und Trennung
\usepackage[latin1]{inputenc}      % Für Umlaute
\usepackage[T1]{fontenc}           % und ß
\usepackage{makeidx}               % Indexpaket
\usepackage{times}                 % Schönere Schriften, Achtung, nicht pslatex verwenden!
\usepackage{longtable}             % Lange Tabelle für Abkürzungsverzeichnis
\usepackage{graphics}              % EPS-Grafiken einbinden
\usepackage{theorem}               % Theorem-Optionen
\usepackage{bbding}
\usepackage{fancyvrb}              % Fancy-Verbatin für Programmlistings
\usepackage{booktabs}
%meine eigenen Pakete
\theoremstyle{break} % Zeilenumbruch nach Theorem
\newtheorem{mydef}{Definition}
%Fürs zeichenen
\usepackage{tikz}
\usetikzlibrary{shapes,arrows}
%Für schöne Algorithmen Darstellungen
\usepackage{listings}
%\usepackage{todonotes}
\usepackage{subfigure}
%- für multirow Tabellen
\usepackage{multirow}
%- für die schöne Darstellung von URLS
\usepackage{url}
%für schöne Formelumgebung
\usepackage{amssymb,amsmath}


\usepackage{boxedminipage}
%Um bei Code Unterschriften zu haben verwendeung von eigenem Listing Stilen (Kai)
%\input{content/listings}
%für mehrsoaltigen Text
\usepackage{multicol}
\usepackage{multirow}
%temp für eigenes comand
% hyperref ohne die Kästchen benutzen, sondern Links dunkelbalu zeigen
\usepackage{color}
\definecolor{lightgray}{gray}{0.9}
\definecolor{myg}{rgb}{213,201,216}
%Für linksiun Dokumenten
\usepackage[breaklinks=true,linkcolor=darkblue,menucolor=darkblue,urlcolor=darkblue]{hyperref}
%\usepackage{breakurl}

%YTe Box zur Interface Beschreibung 
\usepackage{float}
\floatstyle{boxed}
\newfloat{IDL}{thp}{loa}
\floatname{IDL}{Transformation Definition}

%20. 2. 12 Hurenkinder und Schusterjungen verhindern
\clubpenalty = 10000
\widowpenalty = 10000
\displaywidowpenalty = 10000


%%weitere Befhele innerhalb der IDL
\newcommand{\overview}[1]{\textbf{Overview: \\ }{#1}\\}
\newcommand{\arguments}[1]{\textbf{Supplied Arguments: \\ }{#1\\}}
\newcommand{\preCon}[1]{\textbf{Pre-Conditions:  \\}{#1\\}}
\newcommand{\postCon}[1]{\textbf{Results: \\ }{#1\\}}
\newcommand{\error}[1]{\textbf{Errors: \\}{#1\\}}
%eigene Comands für Begriffe:
\newcommand{\automais}{AutoMAIS}
%\newcommand{\staticDataSource}{static data source }
%\newcommand{\staticDataSources}{static data source }
%\newcommand{\dynamicDataSource}{dynamic data source }
%\newcommand{\dynamicDataSources}{dynamic data sources }

% data Connections für datenquellen
% data structuren für datenquellen
% inforamtion für datenquellen 
\newcommand{\information}{information}
\newcommand{\Information}{Information}
% measure
\newcommand{\measure}{measure }
\newcommand{\measures}{measures }
\newcommand{\Measure}{measure }
\newcommand{\Measures}{measures } 
% base measure 
\newcommand{\basemeasure}{base measure }
\newcommand{\basemeasures}{base measures }
\newcommand{\Basemeasure}{base measure }
\newcommand{\Basemeasures}{base measures }


 \newcommand\versicherung[1]{\chapter*{Versicherung}
\thispagestyle{empty}
Hiermit versichere ich, dass ich diese Arbeit selbst\"{a}ndig verfasst und keine anderen als die angegebenen Quellen und Hilfsmittel benutzt habe. Au{\ss}erdem versichere ich, dass ich die allgemeinen Prinzipien wissenschaftlicher Arbeit und Ver\"{o}ffentlichung, wie sie in den Leitlinien guter wissenschaftlicher Praxis der Carl von Ossietzky Universit\"{a}t Oldenburg festgelegt sind, befolgt habe.

\vspace{1.5cm}

#1, den 27. Februar 2012

\vspace*{-0.5cm}\hspace*{5.5cm}\hrulefill\\
\hspace*{6.5cm}\hfill Yvette Teiken \hfill\mbox{}
}


%\newcommand{\RZO}{R$_2$O}
% software engineering
\newcommand{\SE}{software engineering }
\newcommand{\se}{software engineering }

%\newcommand{\awNot}[1]{%
%\begin{boxedminipage}[t]{14cm}
%#1
%\end{boxedminipage}
%   
%}




%Komondo um den Index Eintrag fett zu machen. Benutzt man bei der Definition des Begriffs kommt von RST
\newcommand{\defkey}[1]{\textbf{#1}\index{#1|textbf}}

\title{Dokumentation von BI Projekten} 
%\title{ AutoMais  \\ Automatische Modellgetriebene Analytische Informationssysteme} % Titel der Dissertation
\author{Dr. Yvette Teiken}

                % Tag der Disputation

\makeindex



%Befehl um nur bestimmte Kapitel zu inkludieren um sie zu AW zu geben
%\includeonly{content/evaluation,content/Summary}

% Das eigentliche Dokument
\begin{document}

% Anfang des Dokuments
\maketitle              % Offizielle Titelseite der Uni, bei Abgabe auskommentieren!
%eine leere seite einfügen ohne Seitennummer
%\thispagestyle{empty}
%\cleardoublepage
%\pagenumbering{Roman}   % Römische Seitenzahlen für den Anfang
%\setcounter{page}{5}    % Bei römisch V anfangen	


% Zusammenfassung und Abstract

%\input{content/Dank.tex}
%\input{content/abstract.tex}



% Inhaltsverzeichnis

%\tableofcontents

\cleardoublepage        % Danach auf ungerader Seite weitermachen
\pagenumbering{arabic}  % Arabische Seitenzahlen starten neu

\section{Einleitung}
Dieses Dokument beschreibt wie BI Projekte durchgef�hrt werden sollen. Dieses Dokument stellt Grundlagen f�r die Dokumentation bereit.


\newcommand{\DqBeschreibung}[2]{\textbf{Name der Datenquelle: #1} \\ \textbf{Kurszbeschreibung:} #2 \\ \\}

\newcommand{\Quellattribut}[2]{ \textbf{Attributname}: #1 \textbf{Beschreibung}: #2 \\ }
\section{Datenquellen}

\subsection{Incident Report}
\DqBeschreibung{IncidentReport}{Der Incident Report beschreibt Vorkommen von Fehlerf�llen}
\Quellattribut{ID}{Id von der Gematik geliefert}
\Quellattribut{Beschreibung} {Beschreibung des Incidents im Klartext}


%\Datenbeschreibung{IncidentReort}


\section{Datenintegration}
\subsection{Incident Report}
\subsection{Extraktion}
Hier steht der beschreibne Text, wie das realisiert \\
\textbf{Bsonderheiten: } Hier musste man eine Ausnahme von dem Vorgen in \ref{BP_Extract} gemacht, da hier die Daten anders sind.
\subsection{Kodierung}
\subsection{Struktur}
\subsection{Umsetzung Fachlichkeit} Hier muss die Regel \ref{BP_Extract} Anwendung finden, da ... Realisiert wurde dies mit der SIS Komponenten.

\textbf{Realisiert in Package: } ExtractIncidentReport.dtsx
\section{Zentrale Datenhaltung}
\section{Aggregation}
\section{Reporting}
\section{Fachliche Reglen}
\subsection{Geschlecht zu ICD}\label{GeschlechtZuICD}
\textbf{Fachliche Beschreibung: }wenn ein ICD 40 oder 50 lautet dann kann das Geschlecht nur weiblich sein.\\
\textbf{Dom�nenexperte} Herr Sander \\

\section{Architektur- und Programmierrichtlinien}
In diesem Abschnitt werden Richtlinien und Best Practices gesammelt.

\subsection{Formatierungen}
Es gibt zwei Alternativen hierf�r:
\begin{itemize}
	\item \textbf{Mittels SSIS:} Die Konvertierungen werden mittels SISS Komponenten durchgef�hrt.
	\item \textbf{Mittels SQL:} Die Daten werden als Text in eine DB integriert und dann danach mittels SQL Konvertierungen in das Zielformat �berf�hrt
\end{itemize}


\subsection{Extraktion von Daten} \label{BP_Extract}

%\include{content/Introduction}
% Der eigentliche Inhalt
%Grundlagen
%\include{content/Grundlagen}
%Fallstudie  
%\include{content/ExampleAisCreation}
%\include{content/relatedWork}
%der Ansatz an sich 
%\include{content/autoMAIS}


%\include{content/processModel}

%%achuting hier große Überschrift
%-- Kennzahlen
%\include{content/measures}
% -- Analyse Schema 
%\include{content/AnalyseSchema}
% -- Hierarchien

%\include{content/Hierarchien}

%-- Datenquellen
%\include{content/Datenquellen}


%--Datenquellen Transformationen

%\include{content/DataTransformation}

%--Datenqualität
%\include{content/DatenQualitaet}
%Integrationskapitel
%\include{content/Integration}
%\include{content/architecture}
%\include{content/Transformation}
%\include{content/ConclusionConcept}
%Evaluation
%\include{content/evaluation}
%\include{content/Summary}

%\include{content/Appendix}

% Verzeichnisse am Ende, erst das Glossar
%\include{content/glossar}






% Dann die Abkürzungen
%\include{content/abkuerzungen}

% Weiter mit Abbildungen
%\cleardoublepage
%\addcontentsline{toc}{chapter}{Figures} % Sowohl im Inhaltsverzeichnis als auch als
%\renewcommand{\listfigurename}{Figures} % Überschrift als "Abbildungen" ohne -verzeichnis
%\listoffigures

% Schließlich Literatur
%\cleardoublepage
%\nocite{*}
%\addcontentsline{toc}{chapter}{References} % Soll als "Literatur" auftauchen
%\bibliographystyle{alphadin}                 % Unser Standard-Bibliopgraphiestyle
%24.4. Versuch wegen englischer literatur
%\bibliographystyle{alpha}  
%\renewcommand{\bibname}{References}        % Auch als Überschrift soll "Literatur" erscheinen
%\bibliography{content/bibliographie}            % Literaturverzeichnis einbinden

% Und ganz am Ende der Index
%\cleardoublepage
%\addcontentsline{toc}{chapter}{Index} % Der auch Index heißen soll
%\printindex
%\input{content/index.tex}                     % Index einbinden, vorher aber mit makeindex erzeugen!
%so geht index mit dem Texniccenter "%bm.idx" -o %tm.ind -t %tm.ilg 

%bm.idx" -s olwir.ist -o %tm.ind -t %tm.ilg 


%\versicherung{Oldenburg}
\end{document}

