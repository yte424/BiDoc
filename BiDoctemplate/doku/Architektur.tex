

\newcommand{\DqBeschreibung}[2]{\textbf{Name der Datenquelle: #1} \\ \textbf{Kurszbeschreibung:} #2 \\ \\}

\newcommand{\Quellattribut}[2]{ \textbf{Attributname}: #1 \textbf{Beschreibung}: #2 \\ }
\section{Datenquellen}

\subsection{Incident Report}
\DqBeschreibung{IncidentReport}{Der Incident Report beschreibt Vorkommen von Fehlerf�llen}
\Quellattribut{ID}{Id von der Gematik geliefert}
\Quellattribut{Beschreibung} {Beschreibung des Incidents im Klartext}


%\Datenbeschreibung{IncidentReort}


\section{Datenintegration}
\subsection{Incident Report}
\subsection{Extraktion}
Hier steht der beschreibne Text, wie das realisiert \\
\textbf{Bsonderheiten: } Hier musste man eine Ausnahme von dem Vorgen in \ref{BP_Extract} gemacht, da hier die Daten anders sind.
\subsection{Kodierung}
\subsection{Struktur}
\subsection{Umsetzung Fachlichkeit} Hier muss die Regel \ref{BP_Extract} Anwendung finden, da ... Realisiert wurde dies mit der SIS Komponenten.

\textbf{Realisiert in Package: } ExtractIncidentReport.dtsx
\section{Zentrale Datenhaltung}
\section{Aggregation}
\section{Reporting}
\section{Fachliche Reglen}
\subsection{Geschlecht zu ICD}\label{GeschlechtZuICD}
\textbf{Fachliche Beschreibung: }wenn ein ICD 40 oder 50 lautet dann kann das Geschlecht nur weiblich sein.\\
\textbf{Dom�nenexperte} Herr Sander \\
